% !TeX spellcheck = en_US
\documentclass{amsart}
\usepackage{amsaddr}
 
%%%%%%%%%%%%%%%
%  Help file for the AS\IJ platform
%%%%%%%%%%%%%%% 
\title{Active Segmentation Filters}

\author{Dimiter Prodanov  }
\address{
    PAML-LN,
	IICT, Bulgarian Academy of Sciences,
	Sofia, Bulgaria
}
%\email{dimiter.prodanov@imec.be; dimiterpp@gmail.com}
%%%%%%%%%%%
% Doc
%%%%%%%%%%%
\begin{document}
\date{\today}	
\maketitle
\tableofcontents

%%%%%%%%%%%%
% Section
%%%%%%%%%%%%

\section{Gaussian Scale Spaces}

Smoothing in the digital domain leads to loss of resolution and, therefore, of some information.
The axiomatic linear scale space theory was formulated in series of works by Witkin and Koenderink \cite{Witkin1983, Koenderink1984}.
In its original version, the theory depends on several properties of the Gaussian filters as solutions of the diffusion equation in the scale-space generated by the image.
That is, the generic smoothing kernel $G$ is identified with a radially-symmetric Gaussian kernel of scale $s=\sigma^2 \in \mathbb{R}$
\[
G (r)= \frac{e^{-r^2/2s}}{2 \pi s } = \frac{e^{-(x^2+y^2)/2s}}{2 \pi s }
\]
The Gaussian kernels provide several advantages:  
(i) they are rotationally invariant
(ii) they do not produce artificial extrema in the resulting image
(iii) successive convolutions with different kernels can be combined.
Mathematically, this imposes a very useful semi-group structure, equivalent to the heat/diffusion equation.
In this sense, the image structures diffuse or "melt-down", so that the rate of this diffusion indicates the "robustness" of the structure.

%%%%%%%%%%%%
% Section
%%%%%%%%%%%%
\section{$\alpha$-Scale Spaces}\label{sec:alphascale}

However, the information loss can be limited if one uses multiple smoothing scales 
Pauwels et al. \cite{Pauwels1995} and later Duits et al. \cite{Duits2003} introduced the
$\alpha$-scale spaces for image processing.
The basis of the approach is a generalization the heat equation.
The resulting convolution kernels can be described best by the tools of fractional calculus.
The kernel evolution is governed by two parameters -- the scale \textit{s} and the order of differentiation $\alpha$.
The approach leads to the fractional heat problem:
\begin{flalign*}
	u\left(0, \mathbf{x} \right) &=  I\left( \mathbf{x} \right) \\
	\partial_s	u \left( s, \mathbf{x} \right) & = - (-\Delta)^{\alpha} u\left( s, \mathbf{x} \right) , \quad 0 \leq \alpha \leq 1
\end{flalign*}
where the Riesz fractional Laplacian operator is defined in the Fourier domain  by:
\[
(-\Delta)^{\alpha} U ( \mathbf{k} ) :=  k^{2\alpha} U ( \mathbf{k} ), \quad k= |\mathbf{k}|,
\]
where the $k$ is the modulus of the wave vector $\mathbf{k} $. 
Formally, the operator is extended by continuity for $\alpha=1$ as $ (-\Delta)^{1} = -\Delta \mapsto k^2 $, which corresponds to the usual Laplacian; and to identity for $\alpha=0$, corresponding to invariance in the spatial domain.  
The Green function of the differential equation is the stretched exponential kernel
\[
	G (k, s) =  e^{- k^{2\alpha } s}
\]
in the frequency domain.
%Direct inversion of the Green function gives the Bessel integral
%\[
%	G_{\alpha} (r, s) = \frac{1}{\pi} \int_{0}^{\infty}  J_0( k \,   r)	e^{-  k ^{2\alpha } s}  k \, dk
%\]
%involving the Bessel function of  first kind $J_0(r)$.

%%%%%%%%%%
% Section
%%%%%%%%%%%
\section{Weak differentiation}\label{sec:diff}
Weak differentiation operations in distributional sense can be defined in terms of convolution with the gradient of a  smooth kernel function as:
\[
	\nabla_{G} F: = - F \ast \nabla G 
\]
where the symbol $\nabla$ represents the  of the gradient operator for the Euclidean basis ($e_1$, $e_2$)
\[
\nabla  =  e_1 \frac{\partial}{\partial x} + e_2 \frac{\partial}{\partial y}
\]
Explicitly, using the Gaussian kernel $g = \frac{1}{2 \pi s} e^{-r^2/2s}$
\[
\nabla_{G}= - e_1 \frac{\partial}{\partial x} g - e_2 \frac{\partial}{\partial y} g \equiv - e_1 g_x - e_2 g_y
\] 

%%%%%%%%%%%
% Section
%%%%%%%%%%%
\section{Differential Invariants}\label{sec:diffinv}

There are several types of geometric features useful for image segmentation. 
Typical interesting image features are blobs, filaments and corners. 
The normal Laplacian $\Delta_{ \perp G}$ presented below is sensitive to blobs, while its complement -- the tangential Laplacian $\Delta_{ || G}$ is sensitive to filaments. 


Various geometric features computed by the AS/IJ platform are presented in Table \ref{tab:filters}. 
The normal vector field of the image $F(x,y)$ is defined as
\[
\mathbf n : = \frac{\nabla_G F}{||\nabla_G F ||}
\]
Notable differential invariants are the  amplitude $|\nabla_{G} F|= A$ and the orientation of the gradient, the Hessian determinant, the Hessian eigenvalues, as well as the isophote $\kappa$ and streamline curvatures $\mu$ \cite{Florack1992}.  
Up to a sign convention we have
\begin{flalign}
	\kappa & = \nabla_G \cdot \mathbf{n} \\
	\mu   & =  \nabla_G \cdot \mathbf{t}, \quad  \mathbf{t}=   I_2 \cdot \mathbf{n} 
\end{flalign}
%%%%%%%%%%%%%%%%
where $I_2$ is the pseudoscalar of the Euclidean image plane.
From the perspective of  scale space theory the study of the differential invariants of the image reduce to the study of the differential invariants of the radially symmetric (generalized) heat kernel.

%%%%%%%%%%%%%
%  Table
%%%%%%%%%%%%%
\begin{table*}[pt]
	\centering
	\begin{tabular}{lll}
		\hline
		Filter			   & Functionality & Feature order\\
		\hline
		Gauss2D			   & Gaussian smoothing  & 0 \\
		Gradient		   & Gradient  amplitude and orientation & 1\\
		Gaussian Structure & Structure tensor  & 1  \\
		LoG  			   & Laplacian of Gaussian (LoG) & 2 \\
		ALoG   			   & Anisotropic decomposition of LoG  & 2 \\
		& Gradient amplitude and orientation & 1\\
		Hessian 		   & Eigenvalues of the Hessian & 2 \\
		& Determinant of the Hessian & 2 \\
		Curvature 2D 	   & Line curvatures + Hessian determinant & 2\\
		Curvature 3D 	   & Mean + Gauss curvature of surfaces & 2\\
		BoG   			   & Bi-Laplacian of Gaussian & 4 \\ 
		Gaussian Jet	   & Gaussian Jet of order n & n\\
		LoGN 			   & n-th order PoL  & 2n \\ 
		FFT Kernel LoG	   & Riesz Laplacian  & $\alpha \in \mathbb{R}$ \\
		\hline
	\end{tabular}
	\caption{Filters computing geometric features\cite{Vohra2021}}\label{tab:filters}	
	%	The orientation outputs comprise both the sine and cosine of the phase angle.
\end{table*}
%%%%%%%%%%%%%%%

%%%%%%%%%%%%%
% Sec
%%%%%%%%%%%%%
\section{Orthogonal Laplacian Decomposition (LOD)}\label{sec:lapdecomp}

The simple representation of the gradient introduced above can be extended to a coordinate free (!) operator using the tools of the Geometric Algebra and Calculus. 
Readers are directed to \cite{Macdonald2016} for an introductory material on the subject.
The Laplace operator can be decomposed in two orthogonal components-- on in the direction of the isophote and the other in the direction normal to the isophotes. 
The Laplacian of scalar function \textit{F} can be decomposed into a normal component and tangential component, thus breaking the isotropy of the original operator \cite{Prodanov2022}:
\[
\nabla^2 F = \left( \mathbf{n} \cdot \nabla\right)^2  F + \left(  \mathbf{n} \cdot \nabla F\right)  \nabla \cdot  \mathbf{n} = \Delta_{\perp }F +   \Delta_{|| } F
\]
%where $\mathbf{n}  =\nabla F$ is the gradient, 
where $\mathbf{n}$ is the unit normal vector to the isophote curve $ F (x,y) =c$, 
%\[
%\quad  \mathbf{n}^0 =  {\mathbf{n} }/{|\mathbf{n} |} 
%\]
and $ \mathbf{n} \cdot \nabla$ denotes the directional derivative.
Then also using the isophote curvature the components are given by
\begin{flalign}
	\Delta_{\perp }  &=  \left( \mathbf{n} \cdot \nabla\right)^2 \\
	\Delta_{|| }  &= \left( \nabla \cdot  \mathbf{n}\right)  \left(  \mathbf{n} \cdot \nabla \right) = \kappa \  \mathbf{n}  \cdot \nabla  
\end{flalign}
This is a coordinate-free definition of $\Delta_{\perp}$  and $\Delta_{|| }$, which can be specialized to any smooth coordinate system.
This comes in contrast to the approach of "gauge coordinates" employed in \cite{Florack1992}.
Furthermore, if we specialize to weak Gaussian derivatives
\[
\Delta_{|| G} F= \kappa | A|, \quad |A| =\sqrt{G_x^2+ G_y^2}
\]

%%%%%%%%%%%%%%%%%%%
% Here, $G$ subscript indicates  convolution with a Gaussian kernel or its derivative of an appropriate order.
%Notably,  for a radial kernel function $G(r)$, winch corresponds to spatial isotropy,  and using polar coordinates, the orthogonal decomposition simplifies to
%\[
%\Delta_{\perp} G = \partial_{rr} G, \quad \Delta_{|| } G= \frac{1}{r} \partial_r G
%\]



%%%%%%%%%%%
% Section
%%%%%%%%%%%
\section{Zero-order Geometric Features}
%%%%%%%%%%%
% Section
%%%%%%%%%%%
\subsection{Gauss2D -- Gaussian smoothing}\label{sec:gauss}

The filter convolves an image with the Gaussian kernel in the spatial domain.
\[
G (x,y) \star F (x,y)
\]

%%%%%%%%%%%%%%
% Section
%%%%%%%%%%%%%%
\subsection{FFT Gaussian smoothing}\label{sec:fftgauss}

The filter convolves an image with the Riesz/ Gaussian kernel in the Fourier domain.
  \[
   \mathcal{F}^{-1}\left[  e^{-k^{2 \alpha}} F(k)\right] 
  \]
%%%%%%%%%%%
% Section
%%%%%%%%%%%
\section{First-order Geometric Features}\label{sec:1order}

%%%%%%%%%%%
%  Table
%%%%%%%%%%%
\begin{table}[h!]
	\centering
	\begin{tabular}{ll}
		\hline
		\\
		Gradient amplitude & $ A= \sqrt{G_x^2+ G_y^2} $\\
		Gradient orientation & $ \sin{\phi}= G_y / \sqrt{G_x^2+ G_y^2} $\\
		& $ \cos{\phi}= G_x / \sqrt{G_x^2+ G_y^2} $\\
		\hline
	\end{tabular} 
	\caption{First-order differential invariants}\label{tab:grad} 
\end{table}
%%%%%%%%%%%%%%%%%%%%%%
%%%%%%%%%%%
% Section
%%%%%%%%%%%
\subsection{Gradient }\label{sec:gradient}
The  filter computes the gradient  amplitude and orientation (sine and cosine).
The full output also outputs the elements of the gradient $G_x$ and $G_y$.


%%%%%%%%%%%
% Section
%%%%%%%%%%%
\subsection{Gaussian Structure -- Structure tensor}\label{sec:struct}

The structure tensor (ST) is an abstract extension of the gradient.  
The tensor encodes the predominant directions of the gradient in a specified neighborhood of a point, and the degree to which those directions 
are coherent as a function of scale.
Suppose that we have a scale-space representation of the gradient  $\nabla_G $.
Then the structure tensor is the smoothed tensor product of the smoothed gradient vector\cite{Brox2004}:
\[
S_r (F) := G_r \ast \{ \nabla_G F^T  \nabla_G F    \}
\]
From this expression it is apparent that the operator introduces smoothing on two scales.
However, because of its quadratic characters the scales do not compose.
$S_r (I)$ can be represented by a 2x2 matrix. %, which is positive semidefined.

%%%%%%%%%%%
%  Table
%%%%%%%%%%%
\begin{table}[h!]
	\centering
	\begin{tabular}{ll}
		\hline
		\\
		Laplacian & $\Delta_G= \mathrm{Tr} \, \mathbb{H} = G_{xx} + G_{yy}$ \\
		determinant of the Hessian & $ \mathrm{det} \, \mathbb{H}_G  = G_{xx}  G_{yy} - G_{xy}^2$ \\
		\hline
	\end{tabular} 
	\caption{Second-order differential invariants}\label{tab:det} 
\end{table}
%%%%%%%%%%%%%%%%%%%%%%



%%%%%%%%%%
% Section
%%%%%%%%%%
\section{Second-order Geometric Features} 

%%%%%%%%%%%
% Section
%%%%%%%%%%%
\subsection{LoG -- Laplacian of Gaussian}\label{sec:log}

The Laplacian operator can be thought of as the square of the gradient $\nabla$. 
The analogy can be made precise using the tools of the Clifford algebra and Geometric Calculus:
$
\Delta= \nabla^2
$.
In Cartesian coordinates, the Laplacian has the representation
\[
\Delta_G= G_{xx} + G_{yy}
\]
%where $ \Delta_{G}$ denotes the weak Laplacian operator. 
The filter convolves and image with the Laplacian of Gaussian. 

%%%%%%%%%%%
%%  Sec
%%%%%%%%%%%
%\subsection{FFT LoG} 
%The plugin convolves an image with the Riesz Laplacian kernel of order $\alpha$.
%\[
%\mathcal{F}^{-1}\left[-k^{2 \alpha} e^{-k^{2 \alpha}} F(k)\right] 
%\]
%%%%%%%%%%%
% Section
%%%%%%%%%%%
\subsection{ALoG -- Anisotropic decomposition of LoG}\label{sec:alog}

The theory of the anisotropic decomposition of LoG (ALoG) is given in Sec. \ref{sec:lapdecomp}.
In Cartesian coordinates, the LoG is represented by
\begin{flalign}
	%	\Delta_G & =  \Delta_{||G}  + \Delta_{\perp G} \label{eq:lapd}  \\
	\left( G_x^2+ G_y^2 \right) \Delta_{ \perp G}  & =    \left(  G_x^2 \right)  G_{xx}  +  \left( 2  G_x G_y \right)   G_{xy} +      \left( G_y ^2\right)  G_{yy} \label{ed:lapo}   \\
	\left( G_x^2+ G_y^2 \right) \Delta_{|| G}  &=    \left(  G_x^2\right)  G_{xx}  -   \left( 2 G_x G_y \right)  G_{xy}    +     \left(  G_y ^2\right)  G_{yy}  \label{ed:lapt}    
\end{flalign}
The plugin computes the anisotropic decomposition.
The full output option also outputs the components of the gradient and the Hessian.

%%%%%%%%%%%
% Section
%%%%%%%%%%%
\subsection{Hessian}\label{sec:hessian}

The weak Hessian tensor with respect to the 2\textsuperscript{nd} order derivative of the (Gaussian) kernel \textit{G} is defined as the tensor product
\[
\mathbb{H}_G (F) := \nabla_G \otimes \nabla_G F
\]
%%%%%%%%%%%%%%%%%
For smooth signals, the Hessian is symmetric and can be identified as a metric tensor.
In Cartesian coordinates, the Hessian can be specialized to the usual formula
\[
\mathbb{H}_G (F)=  \left( \begin{array}{ll}
	G_{xx}  & G_{xy} \\
	G_{xy} & G_{yy}
\end{array} \right) \ast F 
\]
where  subscripts denote differentiation by coordinate variables. 
The Hessian generates several  differential invariants.
These are the trace, determinant and the eigenvalues. From now on the notation is abbreviated as $\mathbb{H}_G (F)\equiv \mathbb{H} $.

For the trace holds 
\[
\mathrm{Tr} \,\mathbb{H} = \nabla_G^2 F = \Delta_{G} F
\]
where $ \Delta_{G}$ denotes the weak Laplacian operator.  

The determinant is
\[
\mathrm{det} \mathbb{H}  = G_{xx} G_{yy}- G_{xy}^2
\]
The eigenvalues $\lambda_{1,2}$ can be determined locally from the equation 
\[
	\mathrm{det} (\mathbb{H} - \lambda \mathbb{I} )= 0
\]
This gives the quadratic equation
\[
\lambda^2 - \mathrm{Tr} \,\mathbb{H}  \lambda + 	\mathrm{det} \mathbb{H} =  0
\]
In an explicit form the eigenvalues are
\[
\lambda_{1,2}=\frac{1}{2} \left( (G_{xx} + G_{yy}) \pm \sqrt{(G_{xx} + G_{yy})^2 - 4   \left(  G_{xx} G_{yy}- G_{xy}^2 \right)  } \right) 
\]
The plugin computes the amplitude, sine and cosine of the gradient, the Hessian determinant and its 2 eigenvalues. 
The full output option also outputs the components of the gradient and the Hessian.
%%%%%%%%%%%
% Section
%%%%%%%%%%%
\subsection{Curvature 2D}
The planar image is represented as a set of isophote contours
\[
F(x,y) = c
\]
The gradient vector \textit{n} is orthogonal to the isophote contour.

The plugin computes 2 invariants:
\begin{itemize}
	\item Isophote curvature
	\[  
	%\nabla \cdot \tau = \frac{G_x^2 G_{yy} + G_y^2 G_{xx} }{G_x^2 + G_y^2}
	\kappa= \frac{ G_{xx} G_{y}^2 - 2 G_{x} G_{y} G_{xy} + G_{x}^2 G_{yy}}   {\left(  G_{x}^2 +G_{y}^2\right)^{3/2}}
	\]
	\item Streamline curvature
	\[
	%\mu = \frac{  G_x^2 G_{xy} - G_x G_y G_{xx} - G_{xy} G_y^2 +G_x G_y G_{yy} }   {\left(  G_{x}^2 +G_{y}^2\right)^{3/2}}
	\mu = \frac{ G_x G_y \left(  G_{yy} -  G_{xx} \right) +G_{xy}\left(   G_x^2  -  G_y^2\right)    }   {\left(  G_{x}^2 +G_{y}^2\right)^{3/2}}
	\]
\end{itemize}
The full output option outputs the components of the gradient and the Hessian.

%%%%%%%%%%
%  Section
%%%%%%%%%%
\subsection{Gaussian curvature}
There is a second plugin computing the extrinsic linear curvature 
\[
\nu = \frac{G_{x} G_{yy}- G_{y} G_{xx}}{\left( G_{x}^2 + G_{y}^2\right)^{3/2} }
\]
and the determinant of the Hessian.
The full output option outputs the components of the gradient and the Hessian.




%%%%%%%%%%%
% Section
%%%%%%%%%%%	
\subsection{Curvature 3D}\label{sec:curvature}

The planar image is represented as a surface in the three-dimensional Euclidean space $ \mathbb{E}^3$, where the elevation \textit{z} represents the signal intensity.
\[
z = F(x,y)
\]
%The image intensity generates a surface in 3D.
The plugin computes 
\begin{itemize}
	\item Mean curvature   
	\[
k_m =	\frac{1 }{2}\frac{\left( 1+G_{x}^2\right)  G_{yy} - 2 G_{x} G_{y} G_{xy} +\left(1 +G_{y}^2 \right) G_{xx}  }  {\left( 1 + G_{x}^2 +G_{y}^2\right)^{3/2}} 
	\]
 
	\item Gaussian curvature 
	\[  
k_g=	\frac {G_{xx} G_{yy} - G_{xy}^2}  {\left( 1 + G_{x}^2 +G_{y}^2\right)^2 }
	\]
\end{itemize}
Geometrically, the mean curvature is given by the divergence of the unit normal vector in 3D. 
The full output option outputs the components of the gradient and the Hessian.

%%%%%%%%%%%
% Section
%%%%%%%%%%%	
\subsection{Weingarten Map}\label{sec:weing}
The surface is represented locally at the point P(x,y) by the Monge patch
\[
\gamma = e_1 x +e_2 y + h (x,y) e_3
\]
We define the non-orthogonal un-normalized basis vectors
\[
D_1:= g_x \ast \gamma =e_1 + e_3 g_x \ast h , \quad D_2:= g_y \ast \gamma = e_2 + e_3  g_y \ast h, 
\]
%and further denote $ h_x \equiv   G_x \star h$, $ h_y \equiv   G_y \star h$.

The unit normal to the surface is
\[
n = -I_3 \cdot \frac{D_1 \wedge D_2} {||D_1 \wedge D_2 ||} = \frac{ e_3 - h_x e_1 - h_y e_2 }{\sqrt{ 1+ G_x^2+ G_y^2}} 
\]
The first fundamental form is defined as
\[
I = \begin{pmatrix}
	E & F \\
	F & G
\end{pmatrix}
:= \begin{pmatrix}
	1+ G_x^2 &  G_x G_y \\
	G_x G_y  &  1+ G_y^2 
\end{pmatrix}
\]
or in components
\[
[I]_{ij} = D_i \cdot D_j
\]
The determinant is
\[
\mathrm{det} [I] = 1 + G_x^2 + G_y^2
\]
It is identified with the coefficients of the arc-length differential on the surface patch in surface (u, v) coordinate
\[
ds^2=  [I]_{ij}dx_i dx_j=  E du^2+ 2 F du dv + G dv^2
\]
The first fundamental form encodes the intrinsic geometry of a surface, which is the geometry that can be measured by an inhabitant of the surface without reference to the ambient space.
It allows us to define geometric quantities such as length, angle, and area on the surface using only intrinsic measurements.
The inverse matrix is 
\[
I^{-1} = \frac{1}{1 + G_x^2 + G_y^2} \begin{pmatrix}
	1+ G_x^2 &  -G_x G_y \\
	-G_x G_y  &  1+ G_y^2 
\end{pmatrix}
\]

The second fundamental form is defined as
\[
II = \begin{pmatrix}
	L & M \\
	M & N
\end{pmatrix} :=
\frac{1}{\sqrt{1 +G_x^2 + G_y^2}}
\begin{pmatrix}
	 G_{xx}  &  G_{xy} \\
	 G_{xy}  &  G_{yy} 
\end{pmatrix}
\]
The components are given by
\[
[II]_{ij} = n \cdot  \partial_{i} D_j  % = ((n \wedge e_i) \cdot \nabla) D_j
\]
The second fundamental form describes the deviation of the surface
from its tangent plane.
The second fundamental form measures the normal component of the directional derivative of a tangent vector field along another tangent vector.
It captures how the surface normal changes as the base point is moved along the surface in different directions.
The distance from the surface at r+dr to the tangent plane at r is given by
\[
2 ds^2 = L du^2 + 2 M du dv + N dv^2
\]
%Define $  G_{xy} \star \gamma = D_{xy}$. 
The Weingarten map (shape operator) is defined as
\[
W: = I^{-1} II = \frac{1}{\sqrt{\left( 1 +h_x^2 + h_y^2 \right)^3 }} 
\begin{pmatrix}
	(1+ G_{y}^2) G_{xx} - G_{xy} G_x G_y &  (1+ G_{y}^2) G_{xy} - G_{yy} G_x G_y  \\
	(1+ G_{x}^2) G_{xy} - G_{xx} G_x G_y  & (1+ G_{x}^2) G_{yy} - G_{xy} G_x G_y
\end{pmatrix}
\]
%The components of II give the normal curvature in the direction
The eigenvalues $\lambda_{1,2}$ can be determined locally from the equation 
\[
\mathrm{det} ( {W} - \lambda  {I} )= 0
\]
This gives the quadratic equation
\[
\lambda^2 - \mathrm{Tr} [W]  \lambda + 	\mathrm{det}  [W] =  0
\]
The plugin computes the eignevalues of the Weingarten maps.
They are sensitive to ridge structures.

%%%%%%%%%
%  Section
%%%%%%%%%
\section{Higher order Geometric Features}\label{sec:nroder}

%%%%%%%%%%%%
%  Section
%%%%%%%%%%%%%
%%%%%%%%%
%  Sec
%%%%%%%%%
\subsection{Gaussian Jet}\label{sec:gjet}
In the spatial domain, the  Gaussian derivatives for the one dimensional case can be  computed in closed form as
\[
G_n \left(x \right) = \frac{\partial^n}{\partial x^n} G \left(x \right) = \frac{(-1)^n} {\sqrt{ 2 \pi s^{n+1}}}  \,{He}_{n}\left(  {x}/{\sqrt{s}}\right) \,
e^{-\frac{x^2}{2s}}  
\]
where $He_n (x)$ is the statistician's Hermite polynomial of order \textit{n}.
The sequence of statistician's Hermite polynomials satisfies the recursion
\[
He_{n+1} (x) =x He_{n} (x) - n He_{n-1} (x)    
\]
starting from  $He_0 (x) = 1$ and $He_1 (x) = x$. 
This allows for efficient simultaneous computation of all derivatives up to an order \textit{n} in order to populate the the n-jet space.
%The n-jet components can be used to build the differential invariants up to order \textit{n}. 
The filter computes all Gaussian derivatives up to order \textit{n}.

%%%%%%%%%%%
% Section
%%%%%%%%%%%
\subsection{Bi-Laplacian of Gaussian (BoG)}
The Laplacian operator can be composed multiple times to give rise to the Power-of-Laplacian (PoL) operator \cite{Prodanov2015}:
$
\Delta_G^n I 
$.
The plugin computes $\Delta_G^2 I$.
This operator enhances high-frequency features of an images given the scale cut-off. 
This can be seen easily from the frequency response of the LoG filter.

%%%%%%%%%%%%%%%%%%%%%%
%  Bib
%%%%%%%%%%%%%%%%%
\bibliographystyle{plain}
\bibliography{wekasegmentation}
\end{document}